%%%%%%%%%%%%%%%%%%%%%%%%%%%%%%%%%%%%%%%%%%%%%%%%%%%%%%%%%%%%
%%%%%%%%%%%%%%%%%%%%%%%%%%%%%%%%%%%%%%%%%%%%%%%%%%%%%%%%%%%%
%%%%%%%%%%%%%%%%%%%%%%%%%%%%%%%%%%%%%%%%%%%%%%%%%%%%%%%%%%%%
%%%%%%%%%%%%%%%%%%%%%%%%%%%%%%%%%%%%%%%%%%%%%%%%%%%%%%%%%%%%
%%%%%%%%%%%%%%%%%%%%%%%%%%%%%%%%%%%%%%%%%%%%%%%%%%%%%%%%%%%%
\documentclass[12pt]{article}
\usepackage{epsfig}
\usepackage{times}
\usepackage{amsmath}
\renewcommand{\topfraction}{1.0}
\renewcommand{\bottomfraction}{1.0}
\renewcommand{\textfraction}{0.0}
\setlength {\textwidth}{6.6in}
\hoffset=-1.0in
\oddsidemargin=1.00in
\marginparsep=0.0in
\marginparwidth=0.0in                                               \usepackage{xcolor}                                
\setlength {\textheight}{9.0in}
\voffset=-1.00in
\topmargin=1.0in
\headheight=0.0in
\headsep=0.00in
\footskip=0.50in                                         
\setcounter{page}{1}
\begin{document}
\def\pos{\medskip\quad}
\def\subpos{\smallskip \qquad}
\newfont{\nice}{cmr12 scaled 1250}
\newfont{\name}{cmr12 scaled 1080}
\newfont{\swell}{cmbx12 scaled 800}
%%%%%%%%%%%%%%%%%%%%%%%%%%%%%%%%%%%%%%%%%%%%%%%%%%%%%%%%%%%%
%     DO NOT CHANGE ANYTHING ABOVE THIS LINE
%%%%%%%%%%%%%%%%%%%%%%%%%%%%%%%%%%%%%%%%%%%%%%%%%%%%%%%%%%%%
%     DO NOT CHANGE ANYTHING ABOVE THIS LINE
%%%%%%%%%%%%%%%%%%%%%%%%%%%%%%%%%%%%%%%%%%%%%%%%%%%%%%%%%%%%
%     DO NOT CHANGE ANYTHING ABOVE THIS LINE
%%%%%%%%%%%%%%%%%%%%%%%%%%%%%%%%%%%%%%%%%%%%%%%%%%%%%%%%%%%%

\begin{center}
{\large
PHYSICS  20323/60323: Fall 2023 - LaTEX example 
}\\
%%%%%%%%%%%%%%%%%%%%%%%%%%%%%%%%%%%%%%%%%%%%%%%%%%%%%%%%%%%%
{\large Project: Your Name Here}\\\vskip0.25in
%%%%%%%%%%%%%%%%%%%%%%%%%%%%%%%%%%%%%%%%%%%%%%%%%%%%%%%%%%%%
\end{center}
%%%%%%%%%%%%%%%%%%%%%%%%%%%%%%%%%%%%%%%%%%%%%%%%%%%%%%%%%%%%
% Section Heading
%%%%%%%%%%%%%%%%%%%%%%%%%%%%%%%%%%%%%%%%%%%%%%%%%%%%%%%%%%%%

%%%%%%%%%%%%%%%%%%%%%%%%%%%%%%%%%%%%%%%%%%%%%%%%%%%%%%%%%%%%
% Bullet Point & Numbered list - lists can also be nested as below
%%%%%%%%%%%%%%%%%%%%%%%%%%%%%%%%%%%%%%%%%%%%%%%%%%%%%%%%%%%%


%%%%%%%%%%%%%%%%%%%%%%%%%%%%%%%%%%%%%%%%%%%%%%%%%%%%%%%%%%%%
% Section Heading
%%%%%%%%%%%%%%%%%%%%%%%%%%%%%%%%%%%%%%%%%%%%%%%%%%%%%%%%%%%%
\vskip0.1in

\begin{enumerate}
    \item \textbf{The following questions refer to stars in the Table below.}

Note: There may be multiple answers.

\begin{center}
\begin{tabular}{|l|c|r|r|r|r|}\hline\hline
Name & Mass & Luminosity  & Lifetime & Temperature & Radius \\\hline\hline $\eta$ Car.   & 60. \(M_\odot\) & $10^{6}$ \(L_\odot\) & $8.0 \times 10^{5}$ years & &  \\\hline
$\epsilon$ Eri.   & 6.0 \(M_\odot\) & $10^{3}$ \(L_\odot\) &      & 20,000 K &   \\\hline
$\delta$ Scu.   & 2.0 \(M_\odot\) &         & 5.0 X $10^{8}$ years & & 2 \(R_\odot\) \\\hline
$\beta$ Cyg.   & 1.3 \(M_\odot\) & 3.5 \(L_\odot\)   & & &   \\\hline
$\alpha$ Cen.   & 1.0 \(M_\odot\) &         & & & 1 \(R_\odot\) \\\hline
$\gamma$ Del.   & 0.7 \(M_\odot\) & & 4.5 x $10^{10}$ years & 5000 K &  \\\hline
\end{tabular}\vskip 0.2in
\end{center} 
\begin{enumerate}
    \item (4 points) Which of these stars will produce a planetary nebula.\\

\item (4 points) Elements heavier than \textit{Carbon} will be produced in the stars
\end{enumerate}

2. An electron is found to be on the spin state (in the z-basis): $\chi =   \begin{pmatrix}
    3i\\
    4
    \end{pmatrix}$\\ 
    

(a) (5 points) Determine the possible values of A such that the state is normalized.\\

(b) (5 points) Find the expectation values of the operators \textcolor{red}{$S_x$}, \textcolor{violet}{$S_y$}, \textcolor{orange}{$S_z$} and $\vec{S^2}$.\\


The matrix representations in the z-basis for the components of electron spin operators are
given by:\\

\textcolor{red}{$S_x=\frac{\hbar}{2}\begin{pmatrix}
          0 & 1\\       
          1 & 0
          \end{pmatrix};$} \hspace{10mm}
\textcolor{violet}{$S_y=\frac{\hbar}{2}\begin{pmatrix}
          0 & -i\\       
          i & 0
          \end{pmatrix};$} \hspace{10mm}
\textcolor{orange}{$S_z=\frac{\hbar}{2}\begin{pmatrix}
          1 & 0\\
          0 & -1
          \end{pmatrix}$}\\
          
3. The average electrostatic field in the earth’s atmosphere in fair weather is approximately given:

⃗\begin{equation}
            \vec{E} = E_0(Ae^{\alpha z} + Be^{-\beta z}) \hat{z},    
\end{equation}
              



where \textit{A, B,} $\alpha$, $\beta$ are positive constants and z is the height above the (locally flat) earth surface. \\


(a) (5 points) Find the average charge density in the atmosphere as a function of height \\


(b) (5 points) Find the electric potential as a function height above the earth.

\end{enumerate}

Latex Example \hspace{135mm}31
%%%%%%%%%%%%%%%%%%%%%%%%%%%%%%%%%%%%%%%%%%%%%%%%%%%%%%%%%%%%
\clearpage % inserts a page break
%%%%%%%%%%%%%%%%%%%%%%%%%%%%%%%%%%%%%%%%%%%%%%%%%%%%%%%%%%%%



%%%%%%%%%%%%%%%%%%%%%%%%%%%%%%%%%%%%%%%%%%%%%%%%%%%%%%%%%%%%
% Figures can be inserted
%%%%%%%%%%%%%%%%%%%%%%%%%%%%%%%%%%%%%%%%%%%%%%%%%%%%%%%%%%%%


%%%%%%%%%%%%%%%%%%%%%%%%%%%%%%%%%%%%%%%%%%%%%%%%%%%%%%%%%%%%
% Section Heading
%%%%%%%%%%%%%%%%%%%%%%%%%%%%%%%%%%%%%%%%%%%%%%%%%%%%%%%%%%%%








%%%%%%%%%%%%%%%%%%%%%%%%%%%%%%%%%%%%%%%%%%%%%%%%%%%%%%%%%%%%
% Tables are created easily
%%%%%%%%%%%%%%%%%%%%%%%%%%%%%%%%%%%%%%%%%%%%%%%%%%%%%%%%%%%%


%%%%%%%%%%%%%%%%%%%%%%%%%%%%%%%%%%%%%%%%%%%%%%%%%%%%%%%%%%%%
% Section Heading
%%%%%%%%%%%%%%%%%%%%%%%%%%%%%%%%%%%%%%%%%%%%%%%%%%%%%%%%%%%%



%%%%%%%%%%%%%%%%%%%%%%%%%%%%%%%%%%%%%%%%%%%%%%%%%%%%%%%%%%%%



\end{document}
